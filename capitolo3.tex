


\chapter{App messaggistica in Wi-fi Direct}

\begin{minipage}{12cm}\textit{
    In questo studio 
    di tesi è stata sviluppata un'app per
   dispositivi android che prevede lo scambio di
    messaggi criptati tra due dispositivi,
   nei paragrafi seguenti andremo a illustrarne il 
   funzionamento e analizzeremo
   i principali limiti del Wi-Fi Direct in android.}
\end{minipage}




\section{Wi-Fi Direct in android}



Google ha annunciato il supporto Wi-Fi Direct su Android 4.0 (livello API 14)a ottobre 2011
\cite{wikiDi} e si trova su tutte le versione di android successive.
Le specifiche del Wi-fi Direct, come abbiamo visto nel capitolo 3,
non vietano ad un dispositivo di partecipare a più gruppi contemporaneamente.
Tuttavia allo stato attuale in Android  un dispositivo non
può far parte di due o più gruppi
nello stesso istante di tempo,
di conseguenza se ci si vuole connettere a un'altro gruppo
bisogna disconnettersi da quello a cui si è connessi, riavviare la fase di discovery
e connettersi a gruppo desiderato.

Il Wi-Fi P2P su Android a livello implementativo, è composto da un insieme di
funzioni disponibili al programmatore, chiamate API e che sono divise in tre parti
principali:

\begin{figure}
   \centering
   \caption{metodi principali della classe WifiP2pManager}
   \includegraphics[width=1\columnwidth]{imgs/wifip2pmanagerMet.jpg}
\end{figure}

\begin{itemize}
   \item tutti i metodi definiti nella classe WifiP2pManager che
   consentono la scansione,
   richiesta e connessione con i peer;
   \item i listener, “oggetti” che permettono di notificare
   stati di “successo” o
   “fallimento” quando si eseguono metodi della classe WifiP2pManager;
   \item Intent che notificano ogni
   specifico evento rilevato dal device, ad
   esempio se la connessione cade, se un peer è stato appena scoperto ecc.
\end{itemize}
Qui di seguito in figura, sono esposti i metodi della classe WifiP2pManager.
Ogni metodo della classe WifiP2pManager è collegato a un listener, il quale a
seconda dell’esito della chiamata del metodo, si occupa di avvisare con un
messaggio di successo o fallimento.
Inoltre le API del Wi-Fi P2P definiscono degli intent che, registrati su un
BroadcastReceiver, permettono all’applicazione di rilevare gli eventi che
accadono in un preciso istante.
\begin{figure}
   \centering
   \caption{lista dei listener associati ai vari metodi della classe WifiP2pManager}
   \includegraphics[width=1\columnwidth]{imgs/listenerwifip2pmanager.jpg}
\end{figure}




\section{Funzionamento}
\begin{figure}
   \centering
   \caption{inizio fase di scan}
   \includegraphics[width=1\columnwidth]{imgs/discoveryStart.png} % Example image
\end{figure}
ho utilizzato le API di Android Software Development Kit (SDK)
in Android Studio \cite{ASD} per lo sviluppo di questa applicazione
\subsection{Descrizione alto livello}


L'applicazione prevede la comunicazione tra due dispositivi android,
quest'ultimi entrano nella fase di discovery e una volta che si sono trovati
possono instaurare una connessione,
che una volta stabilita aprirà un canale di comunicazione bidirezionale tra i due.
I due dispositivi adesso possono scambiare i messaggi.


\subsection{Fase di scan}
\begin{figure}
   \caption{listener dei peer}
   \centering
   \includegraphics[width=1\columnwidth]{imgs/peerListener.png}% Example image
\end{figure}


per iniziare la fase di scan si utilizza mManager, un
oggetto che abbiamo istanziato nella MainActivity
di tipo WifiP2pManager,questa classe ci fornisce le API per gestire le connessioni
Wi-Fi P2P \cite{androiddevelopers},
come possiamo vedere dalla figura 3.3.
Per ricavare la lista dei peer (dispositivi vicini)
si utilizza un listener,
che ogni volta rileva un nuovo peer lo salva nel suo database
interno al dispositivo e lo
visualizza su schermo insieme agli altri, il codice è mostrato
in figura 3.4


\subsection{Instaurazione della connessione e selezione del Go}



Una volta che l'utente ha su schermo la lista dei dispositivi
vicini può scegliere il dispositivo con il quale comunicare
semplicemente cliccandoci sopra il dispositivo proverà a
connettersi al device selezionato usando il metodo "connect"
della classe "WifiP2pManager" si vede dalla figura 3.5.
\begin{figure}
   \caption{richiesta connessione}
   \includegraphics[width=1\columnwidth]{imgs/Connect.png}
\end{figure}

Dopo di che entra in gioco il listener della connessione.
In quest'ultimo verrà scelta la classe da avviare
(client o server)in base al dispositivo che è
diventato GO come si vede nel codice in figura 3.6.

\begin{figure}
   \caption{}
   \includegraphics[width=1\columnwidth]{imgs/listenerConeessione.png}
\end{figure}

\subsection{Scambio di messaggi}
\subsubsection{Server Class}
Nel caso al dispositivo assume il ruolo del GO
viene istanziato un oggetto della  classe ServerClass,
che accetta una connessione sulla porta 8888
attraverso una serverSocket che sarà utilizzata per comunicare
attraverso la classe SendReceive come vedremo più avanti

\begin{figure}
   \caption{server class}
   \includegraphics[width=0.8\columnwidth]{imgs/serverClass.png}
   \centering
\end{figure}


\subsubsection{Client Class}
Invece nel caso il dispositivo non assume il ruolo del GO
viene istanziato un oggetto della classe ClientClass,
che prova a connettersi all'indirizzo del GO sulla porta 8888 attraverso
una socket che sarà utilizzata per comunicare
attraverso la classe SendReceive come vedremo più avanti.
  
\begin{figure}
   \caption{client class}
   \includegraphics[width=1\columnwidth]{imgs/clientClass.png}
   \centering
\end{figure}


\subsubsection{Scambio di messaggi attraverso SendReceive}
SendReceive si occupa di inviare i messaggi all'altro dispositivo
e di riceverli;
per inviarli utilizza il metodo write che scrive
sull'output stream della socket
per riceverli controlla continuamente l'input stream
della socket


\subsection{Crittografia usata}
Sebbene Wi-Fi Direct utilizza  che usa una crittografia di tipo
AES \cite{Wi-FiProtected}
nell'app si è voluto aggiungere un ulteriore layer di crittografia
implementato a livello software.
Per criptare i messaggi è stata usata la libreria
Spongy Castle \cite{Spongy} questa libreria è stata derivata
da Bouncy Castle in quanto 
la piattaforma Android sfortunatamente viene
distribuita con una versione ridotta di Bouncy Castle,
Spongy Castle è uguale a Bouncy Castle
ma con un paio di piccole modifiche per farlo funzionare su Android.
per generare la coppia di chiavi ho usato la curva ellittica con parametri
(Standards for Efficient Cryptography)
"secp224k1" \cite{sec}.
Per gestire la crittografia è stata creata una classe Cript che contiene
gli oggetti e i metodi necessari per essa: la coppia di chiavi pubblica e privata
del dispositivo e la chiave pubblica dell'altro dispositivo; per quanto riguarda i metodi
invece ce ne sono 4:
\begin{itemize}
   \item Cript() è il costruttore della classe Cript e
   inizializza la coppia di chiavi pubblica e privata.
   \begin{figure}
       \caption{costruttore della classe Cript}
       \includegraphics[width=1\columnwidth]{imgs/Criptconstructor.jpg}
   \end{figure}

   \item setHisKey() prende in input la chiave pubblica dell'altro dispositivo
   encodata in base64 ne fa il decode la memorizza all'interno dell'oggetto
   nel campo "hispub".
   \begin{figure}
       \caption{setter del campo hispub}
       \includegraphics[width=1  \columnwidth]{imgs/sethiskey.jpg}
   \end{figure}



   \item encript() prende in input il messaggio da criptare e lo cripta con la
   chiave pubblica dell'altro dispositivo e ritorna il messaggio criptato
   encodato in base64.
   \begin{figure}
       \caption{metodo per criptare il messaggio}
       \includegraphics[width=0.8  \columnwidth]{imgs/encript.jpg}
   \end{figure}

   \item decript() prende in input il messaggio da decriptare encodato in base64
   e lo decripta con
   la sua chiave privata e ritorna il messaggio decriptato.
   \begin{figure}
       \caption{metodo per decriptare il messaggio}
       \includegraphics[width=1  \columnwidth]{imgs/decript.jpg}
   \end{figure}



\end{itemize}

\subsubsection{scambio chiavi pubbliche dei dispositivi}
Una volta che I dispositivi si sono connessi e hanno istanziato
rispettivamente la classe server e client sono pronti a scambiarsi i messaggi
e il primo messaggio che si scambiano in modo automatico è la loro chiave pubblica
questo senza che l'utente si accorga di nulla.
la coppia di chiavi pubblica e privata vengono generate a ogni nuova connessione.


\section{Analisi del Wi-Fi Direct}

Dall'esperienza acquisita sviluppando questo prototipo di app è emerso
che il Wi-Fi Direct risulta ottimo per connettere i dispositivi singolarmente
fornendo velocità di trasmissione standard del wifi questa cosa è stata confermata
anche dai test effettuati sul campo che mostreremo più avanti,
mentre nonè adatto a formare una rete di dispositivi connessi in quanto
come già spiegato in precedenza in android non è permessa la connessione
del dispositivo a due Gruppi differenti nello stesso momento.
Dai test si è osservato che il tempo di discovery e quello di formazione del gruppo
hanno mostrato risultati discordanti in quanto con l'aumento della distanza quest'ultimi
impiegavano meno tempo, si noti però che durante lo sviluppo dell'app di messaggistica
in alcuni casi la fase di discovery è arrivata a richiedere anche più 10 secondi.
I test che ho fatto inoltre hanno mostrato il limite della portata
del Wi-Fi Direct si è osservato che fino a 64 metri
(all'aperto e senza ostacoli) i dispositivi
si connettevano e riuscivano a scambiare messaggi  mentre a 128 metri
mentri sebbene i disponibili si vedessero (con difficoltà) non riuscivano
a stabilire una connessione non riuscendo quindi neanche a scambiarsi messaggi,
per questo motivo i test a questa distanza non sono stati riportati nella tabella.
Di seguito riporteremo i vari test sulle varie distanze con i tempi (in secondi)di:
discover,group formation,e trasferimento di un payload da 10MByte.
Un altro limite del Wi-Fi Direct è il consumo di una batteria che risulta
essere elevato.

\subsubsection{I test effettuati}
I test sono stati effettuati attraverso un'app sviluppata da me
su due dispositivi rispettivamente,
uno xiaomi redmi 5 pro e uno xiaomi redmi note 6, pro dove tengo
traccia rispettivamente del tempo che i due dispositivi impiegano a
trovarsi, il tempo di formazione del gruppo e il tempo che impiegano
per inviare una payload di 10 MByte.
I test sono stati eseguiti per le seguenti distanze in metri 0,4,8,16,64,128
per ogni distanza il test è stato eseguito 5 volte
i valori che riporteremo sono la risultante della media dei 5.
il test viene eseguito nel seguente: modo chiameremo d1 il dispositivo 1
e d2 il dispositivo 2,
d2 entra per primo in modalità discovery dopo di che anche d1 entra in modalità
discovery adesso il tempo che d1 impiega per trovare d2 sarà il tempo
di discovery che viene registrato. Per tempo di formazione del
gruppo si intende il tempo che uno dei due dispositivi impiega a
diventare Group owner, una volta che i dispositivi sono connessi
si procede con l'invio del payload e registrazione del tempo impiegato.

%https://www.tablesgenerator.com/latex_tables
\begin{table}
   \centering
   \begin{tabular}{|c|c|c|c|}
   \hline
   \begin{tabular}[c]{@{}c@{}}Distanze\\ in\\ metri\end{tabular} & \begin{tabular}[c]{@{}c@{}}Tempo \\ (in secondi)\\ Discovery\end{tabular} & \begin{tabular}[c]{@{}c@{}}Tempo\\ (in secondi)\\ formazione\\ gruppo\end{tabular} & \begin{tabular}[c]{@{}c@{}}Tempo\\ (in secondi)\\ invio\\ payload \\ di 10 Mbyte\end{tabular} \\ \hline
   0                                                             & 2.42                                                                      & 1.84                                                                               & 2.5                                                                                           \\ \hline
   4                                                             & 1.45                                                                      & 1.27                                                                               & 5.25                                                                                          \\ \hline
   8                                                             & 1.04                                                                      & 0.94                                                                               & 5.8                                                                                           \\ \hline
   16                                                            & 0.1                                                                       & 1.44                                                                               & 12.4                                                                                          \\ \hline
   32                                                            & 4,6                                                                       & 1.6                                                                                & 14,6                                                                                          \\ \hline
   64                                                            & 1,12                                                                      & 1.42                                                                               & 17,6                                                                                          \\ \hline
   \end{tabular}
   \end{table}

   \chapter*{Conclusioni}
   In questa tesi inizialmente si è visto cosa è una Near-Me area network e
   si è fatta una panoramica sulle reti peer to peer come si correlassero 
   con il Wi-fi Direct; successivamente siamo andati ad approfondire le specifiche
   dello standard del Wi-Fi Direct andando ad analizzarne il
    funzionamento vero e proprio.
   Dopo questa parte si è passati allo studio del Wi-Fi Direct
   in android e la spiegazione dell'implementazione
   proposta di un'app , sviluppata per questo studio di tesi,
   che usa il Wi-Fi Direct per lo scambio di messaggi
   tra due dispositivi.
   Successivamente siamo andati a testare le performance del Wi-Fi Direct
   attraverso un'altra app sviluppata appositamente per lo scopo.
   Da questo studio di tesi è emerso che allo stato attuale
   non è possibile costruire una rete peer to peer
   attraverso il Wi-Fi Direct ma più tosto è più orientato
   per connessioni peer to peer.







