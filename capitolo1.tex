\chapter{Panoramica Generale}


\section{Near-Me Area Network}

Internet utilizza diversi tipi di reti di comunicazione. Una rete
locale (LAN) copre una piccola area geografica, come una scuola o
un'azienda; una rete area metropolitana (MAN) di solito si estende su
un'area più ampia, come una città o uno stato, mentre una rete geografica
(WAN) fornisce comunicazioni in un'ampia area geografica che copre posizioni
nazionali e internazionali. Le reti personali (PAN) sono LAN wireless con una
portata molto breve (fino a pochi metri), che consente ai dispositivi del
computer (come PDA e stampanti) di comunicare con altri dispositivi e computer
vicini. A causa della crescente popolarità dei dispositivi mobili abilitati
alla localizzazione, sta emergendo un nuovo tipo di rete di comunicazione: la
NAN (Near-Me Area Network).
\subsection{Cos'è un NAN?}
Un NAN è una rete di comunicazione costruita su infrastrutture di rete
fisica esistenti che si concentra sulla comunicazione tra dispositivi wireless
nelle immediate vicinanze. A differenza delle LAN, in cui i dispositivi si
trovano nello stesso segmento di rete e condividono lo stesso dominio di
trasmissione, i dispositivi in ​​una NAN possono appartenere a diverse
infrastrutture di rete proprietarie (ad esempio, diversi operatori di telefonia
mobile). Quindi, anche se due dispositivi sono geograficamente vicini, il
percorso di comunicazione tra loro potrebbe, infatti, attraversare una lunga
distanza, passando da una LAN, attraverso Internet, e ad un'altra LAN.

Sebbene i dispositivi mobili abbiano fornito servizi di localizzazione da
molto tempo. Il concetto di NAN e le loro applicazioni sono emersi solo di
recente. Utilizzando la
posizione geografica dei dispositivi mobili, gli utenti possono accedere a
informazioni specifiche sulla loro posizione, la posizione di bancomat o delle
stazione di
servizio più vicine. Tali servizi si concentrano sull'accesso di un utente alle
informazioni, mentre le applicazioni NAN si concentrano sulle comunicazioni a
due vie tra le persone che si trovano in una certa prossimità l'una dell'altra.
D'altro canto, le applicazioni NAN non sono sempre interessate alle posizioni
esatte di quelle persone.



\section{Reti Peer-To-Peer (P2P)}
L'architettura peer-to-peer è un tipo di rete in cui ogni nodo ha capacità e
responsabilità equivalenti. Questo differisce dalle architetture client /
server in cui alcuni nodi sono dedicati a servire gli altri. Le reti
peer-to-peer sono generalmente più semplici ma in genere non offrono le stesse
prestazioni in presenza di carichi pesanti. La stessa rete P2P fa affidamento
sulla potenza di calcolo alle estremità di una connessione anziché all'interno
della stessa rete.

Il P2P viene spesso erroneamente utilizzato come termine per descrivere un
utente che si collega con un altro utente per trasferire informazioni e file
attraverso l'uso di un client P2P comune per scaricare MP3, video, immagini,
giochi e altri software. Questo, tuttavia, è solo un tipo di rete P2P.
Generalmente, le reti P2P sono utilizzate per condividere file, ma una rete P2P
può anche significare Grid Computing o Instant messaging.

\subsection{Tipi di reti P2P}
\subsubsection{Collaborative Computing}
Definito anche calcolo distribuito, combina la potenza di elaborazione inattiva
o inutilizzata della CPU e / o lo spazio libero su disco di molti computer
nella rete. Il calcolo collaborativo è più popolare con le organizzazioni
scientifiche e biotecnologiche in cui è richiesta un'intensa elaborazione del
computer.
\subsubsection{Instant Messaging}
Una forma molto comune di networking P2P è Instant Messaging (IM) in cui le
applicazioni software, come MSN Messenger o AOL Instant Messenger,
consentono agli utenti di chattare tramite messaggi di testo in tempo reale.
Mentre la maggior parte dei venditori offre una versione gratuita del proprio
software di messaggistica istantanea, altri hanno iniziato a concentrarsi sulle
versioni enterprise del software di messaggistica istantanea, mentre le aziende
si sono mosse verso l'implementazione di messaggistica istantanea come
strumento di comunicazione standard per le aziende.

\subsubsection{Peer-to-peer File-sharing}
Una volta scaricato e installato un client P2P, se si è connessi a Internet è
possibile avviare l'utilità e accedere a un server di indicizzazione centrale.
Questo server centrale indicizza tutti gli utenti che sono attualmente connessi
online al server. Questo server non ospita file da scaricare. Il client P2P
conterrà un'area in cui è possibile cercare un file specifico. L'utility
interroga il server di indicizzazione per trovare altri utenti connessi con il
file che si sta cercando. Quando viene trovata una corrispondenza, il server
centrale ti dirà dove trovare il file richiesto. È quindi possibile scegliere
un risultato dalla query di ricerca e dall'utilità quando si tenta di stabilire
una connessione con il computer che ospita il file richiesto. Se viene
stabilita una connessione, inizierai a scaricare il file. Una volta completato
il download del file, la connessione verrà interrotta.
Un secondo modello di client P2P funziona allo stesso modo ma senza un server
di indicizzazione centrale. In questo scenario, il software P2P cerca
semplicemente altri utenti di Internet utilizzando lo stesso programma e li
informa della tua presenza online, costruendo una vasta rete di computer man
mano che più utenti installano e utilizzano il software.

\subsection{Wi-Fi Direct per le reti peer to peer}

Il Wi-Fi Direct anche chiamato Wi-fi peer to peer è una tecnologia
relativamente nuova.
Vendendo le specifiche teoricamente sarebbe possibile creare una rete
peer to peer in quanto un dispositivo si potrebbe connettere a due
gruppi differenti (nello stesso momento) ed andare a formare una
vera e propria rete,
ma in android (alla versione attuale 9) questa cosa non è supportata
quindi per aggirare questa limitazione un dispositivo si dovrebbe
connettere prima a un gruppo "A" poi nel momento in cui si vuole
connettere a un gruppo "B"
si deve disconnettere da "A" e connettersi a "B".
Il prototipo di scambio di messaggi proposto in questo studio di
tesi prevede lo scambio di messaggi tra due dispositivi "A" e "B"
nello stesso istante di
tempo se il dispositivo "A" vuole scambiare messaggi con un altro dispositivo "C"
si deve disconnettere da "B" e connettersi a "C".
Nel capitolo seguente andremo ad analizzare le specifiche del Wi-Fi Direct.


