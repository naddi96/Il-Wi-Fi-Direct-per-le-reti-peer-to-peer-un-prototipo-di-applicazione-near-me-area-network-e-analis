\chapter*{Ringraziamenti}
\addcontentsline{toc}{chapter}{Ringraziamenti}

Se lo si desidera, inserire qui un breve elenco di ringraziamenti riguardo la tesi.\\

Non superare possibilmente la lunghezza di una pagina!


\chapter*{Introduzione}
\addcontentsline{toc}{chapter}{Introduzione}


Nel 2017 sono stati contati  223 milioni di utenti Smartphone solo negli U.S.A.
\cite{statista:1}. Questi numeri impressionanti stanno crescendo senza freni e sono destinati
solamente ad aumentare, infatti si prevedono
247.5 milioni di utenti Smartphone nel 2019 \cite{statista:1}.
Inoltre, abbiamo visto l'affermazione del mercato di dispositivi “Smart”, come per esempio
quello degli Smart Watch, che sta esplodendo, o la
diffusione di dispositivi di fascia bassa nei paesi in via di sviluppo.
Con il crescere del numero degli utenti, stanno aumentando anche il numero delle
funzionalità di ogni dispositivo, sempre più vicine ad eguagliare quelle dei classici
computer portatili. Anche le richieste e le esigenze degli utenti stessi stanno crescendo,
alimentando nuove ricerche. Un esempio è il desiderio di essere sempre connessi
ad internet e poter navigare a velocità sempre più elevate in ogni luogo. Infatti, in
precedenza il problema della connettività su dispositivi mobili non è mai stato affrontato
nel modo in cui si è costretti a fare ora. Adesso vi è una vera e propria spinta
e richiesta degli utenti che dovrà essere soddisfatta. La comunicazione come siamo
abituati a concepirla non è più sufficiente, ora si vuole un vero e proprio scambio di
grandi quantità di dati in mobilità. Il concetto di “mobilità” non è stato considerato
nel modo in cui lo vediamo oggi, perché alla fine degli anni ’90, quando è nata la
tecnologia Wi-Fi, è stata pensata per un ambiente più statico. Oggi invece gli utenti
richiedono sempre più dinamismo e di conseguenza dovrà, e di fatto sta subendo,
una vera e propria evoluzione.
Questi desideri stanno dando vita ad un mondo, che negli ultimi anni è stato
definito Internet Of Things, cioè un nuovo modo di usare la rete, dove ogni oggetto
è connesso. In questo settore possiamo considerare il concetto di Opportunistic
Networks, cioè reti senza fili, in cui i nodi sono dispositivi portati da utenti, senza
infrastrutture di rete, con ricerca e comunicazione automatica in ambienti vari ed
estremamente dinamici. In queste reti l’utente e il dispositivo sono un tutt’uno.
Queste nuove richieste si sono scontrate con un settore non pronto per soddisfarle,
cioè le tecnologie e protocolli di comunicazione, che hanno dovuto reinventare la
comunicazione tra dispositivi mobili con requisiti differenti. Infatti, negli ultimi anni
stanno nascendo nuove soluzioni che mirano a rendere tali scenari non più idee in
ambito della ricerca, ma pura realtà alla portata di tutti. Oggi ci sono diverse di
queste tecnologie, ma sono ancora molto giovani, come per esempio Wi-Fi Direct,
Bluetooth 5 ed LTE Direct (ancora non disponibile pubblicamente).


\section{Scopo del lavoro}

In questo lavoro di tesi mi concentrerò sui progressi
che sono stati fatti sia dal punto di vista teorico,
ma soprattutto pratico, confrontando i due.
Nello specifico sul Wi-Fi Direct analizzandone i limiti e attraverso lo sviluppo
di un'app di messaggistica criptata che ne fa uso.
Il problema principale per il suo utilizzo in scenari per lo
più assimilabili ad Opportunistic Networks, è che il protocollo non è stato pensato per
gestire reti in situazioni molto dinamiche ed imprevedibili. Lo stato attuale di questo
protocollo di rete è più vicino all’essere un passo intermedio verso il raggiungimento
dell’obiettivo delle Opportunistic Networks.




